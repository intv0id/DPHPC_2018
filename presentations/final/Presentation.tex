%%% Décommenter pour une présentation
\documentclass{beamer}
%%%

%%% Décommenter pour avoir un article
%\documentclass[handout]{beamer}
%\usepackage{pgfpages}
%\pgfpagesuselayout{2 on 1}[a4paper,border shrink=5mm]
%\setbeameroption{show notes on second screen=bottom} % Beamer manual, section 19.3
%%%

%%%%%%%%%%%%%%%%%%%%%%%%%%%%%%%%%%%%%%%%%%%%%%%%%%%%%%%%%%%%%%%%%%%%%%
% NE PAS MODIFIER CETTE PARTIE

\usetheme{Warsaw}
\usepackage[utf8]{inputenc}
\usepackage[T1]{fontenc}
\usepackage{amsmath}
\usepackage{amsfonts}
\usepackage{amssymb}
\usepackage{graphicx}
\usepackage{listings}
\usepackage{hyperref}
\usepackage{tikz}
\usepackage{verbatim}
\usepackage{algorithm}
\usepackage{algorithmic}
\usetikzlibrary{arrows,shapes}

\graphicspath{{graphics/}}

\hypersetup{pdfstartview={Fit}}



\AtBeginSection[]{
  \begin{frame}
  \vfill
  \centering
  \begin{beamercolorbox}[sep=8pt,center,shadow=true,rounded=true]{title}
    \usebeamerfont{title}\insertsectionhead\par%
  \end{beamercolorbox}
  \vfill
  \end{frame}
}

\setbeamertemplate{note page}[plain] % Beamer manual, section 19.1
\newlength{\parskipbackup}
\setlength{\parskipbackup}{\parskip}
\newlength{\parindentbackup}
\setlength{\parindentbackup}{\parindent}
\newcommand{\baselinestretchbackup}{\baselinestretch}
\usetemplatenote{\rmfamily \scriptsize%
  \setlength{\parindent}{1em} \setlength{\parskip}{1ex}%
  \renewcommand{\baselinestretch}{1}%
  \noindent \insertnote%

  \setlength{\parskip}{\parskipbackup}%
  \setlength{\parindent}{\parindentbackup}%
  \renewcommand{\baselinestretch}{\baselinestretchbackup}%
}

\logo{
	\includegraphics[scale=0.1]{SPCL.png}
} 
\institute[ETH Zürich]{\textbf{ETH Zürich}}




%%%%%%%%%%%%%%%%%%%%%%%%%%%%%%%%%%%%%%%%%%%%%%%%%%%%%%%%%%%%%%%%%%%%


%TODO : PARTIE A MODIFIER 

\date{December 2018}	

\author{
    Th. Cambier
    R. Dang-Nhu
    Th. Dardinier
    C. Trassoudaine
}
\title{
	\textbf{Minimum Spanning Tree}\\
	-\\ 
	\textit{DPHPC}
}

\usepackage[backend=biber, style=authoryear, doi=false,isbn=false,url=false, giveninits=true]{biblatex}
\bibliography{bib.bib}
 

\begin{document}

\pgfdeclarelayer{background}
\pgfsetlayers{background,main}

\frame{\titlepage}
\frame{\tableofcontents}

%%%%%%%%%%%%%%%%%%%%%%%%%%%%%%%%%%%%%%%%%%%%%%%%%%%%%%%%%%%%%%%%%%%%
% SECTION 1
%%%%%%%%%%%%%%%%%%%%%%%%%%%%%%%%%%%%%%%%%%%%%%%%%%%%%%%%%%%%%%%%%%%%


\section{Problem definition - reminder}
\subsection{The MST Problem}
\begin{frame}
\frametitle{The MST problem}
 A minimum spanning tree (MST) or minimum weight spanning tree is a subset of the edges of a connected, edge-weighted (un)directed graph that connects all the vertices together, without any cycles and with the minimum possible total edge weight. 
 \includegraphics[width=.5\textwidth]{MST.png}
\end{frame}


\subsection{Use cases}

\begin{frame}
\frametitle{Input sets: $G(n,p)$}
\centering
$G(100, 0.02)$
\begin{figure}
 \includegraphics[width=.7\textwidth]{graphGNP.png}
\end{figure}
\end{frame}

\begin{frame}
\frametitle{Input sets: $PA(n)$}
\centering
$PA(100)$
\begin{figure}
 \includegraphics[width=.7\textwidth]{graphPA.png}
\end{figure}
\end{frame}

\begin{frame}
\frametitle{Input sets: 9\textsuperscript{th} DIMACS challenge dataset}
\centering
USA Roads
\begin{figure}
 \includegraphics[width=.7\textwidth]{graphUSA.png}
\end{figure}
\end{frame}

%%%%%%%%%%%%%%%%%%%%%%%%%%%%%%%%%%%%%%%%%%%%%%%%%%%%%%%%%%%%%%%%%%%%
% SECTION 2
%%%%%%%%%%%%%%%%%%%%%%%%%%%%%%%%%%%%%%%%%%%%%%%%%%%%%%%%%%%%%%%%%%%%

\section{Setup}

\begin{frame}
\frametitle{Software}
\centering

\begin{itemize}
\item OMP
\item Intel Threading Building Blocks (TBB)
\item Parallel Streaming Transformation Loader Service (PSTL)
\end{itemize}

\end{frame}

\subsection{Hardware}

\begin{frame}
\frametitle{EULER}
\centering

\begin{itemize}
\item 1 node limitation (OMP)
\item 2 sockets filled with 18 cores - up to 3.7Ghz
\item Inter-sockets bus speed: 10.4 GT/s
\end{itemize}

\includegraphics[width=5cm]{dual_sockets_caches.png}
\end{frame}


%%%%%%%%%%%%%%%%%%%%%%%%%%%%%%%%%%%%%%%%%%%%%%%%%%%%%%%%%%%%%%%%%%%%
% SECTION 3
%%%%%%%%%%%%%%%%%%%%%%%%%%%%%%%%%%%%%%%%%%%%%%%%%%%%%%%%%%%%%%%%%%%%

\section{Algorithms and parallel implementations}

\subsection{Base serial algorithms}

\begin{frame}[fragile]
\frametitle{Sollin}
\small
\begin{algorithm}[H]
\begin{algorithmic}[1]

\STATE F = set(one-vertex trees)
\WHILE{$\mid F \mid > 1$}
\STATE TODO
\ENDWHILE

\end{algorithmic}
\end{algorithm}
\end{frame}

\begin{frame}[fragile]
\frametitle{Kruskal}
\small
\begin{algorithm}[H]
\begin{algorithmic}[1]
\STATE $A = \emptyset$
\FORALL{$v \in G.V$}
\STATE MAKE-SET$(v)$
\ENDFOR
\STATE Sort (asc.) $\left(weight(u, v)\right)_{(u, v) \in G.E}$
\FORALL{$(u, v)$ in $G.E$ ordered by weight}
\IF{FIND-SET$(u)$ $\neq$ FIND-SET$(v)$}
\STATE $A = A \cup {(u, v)}$
\STATE UNION$(u, v)$
\ENDIF
\ENDFOR
\RETURN A
\end{algorithmic}
\end{algorithm}


\end{frame}

\begin{frame}
\frametitle{Boost implementations}

\begin{columns}
\begin{column}{.7\linewidth}
\begin{itemize}
\item Boost-Kruskal used as a reference
\end{itemize}
\end{column}

\begin{column}{.3\linewidth}
\includegraphics[width=\linewidth]{boost.png}
\end{column}
\end{columns}

\end{frame}


%%%%%%%%%%%%%%%%%%%%%%%%%%%%%%%%%%%%%%%%%%%%%%%%%%%%%%%%%%%%%%%%%%%%
% SECTION 4
%%%%%%%%%%%%%%%%%%%%%%%%%%%%%%%%%%%%%%%%%%%%%%%%%%%%%%%%%%%%%%%%%%%%

\subsection{Parallel improvements}

\begin{frame}
\frametitle{Parallel sorting on Kruskal}

\end{frame}

\begin{frame}
\frametitle{Filter Kruskal}

\end{frame}

\begin{frame}
\frametitle{Filter Sollin}

\end{frame}


\begin{frame}
    \printbibliography
\end{frame} 

\section{Results overview}

\subsection{Setup}

\begin{frame}
\frametitle{EULER Cluster}

\end{frame}

\subsection{Results}

\begin{frame}
\frametitle{Scalability}

\end{frame}

\begin{frame}
\frametitle{Speedups}

\end{frame}




\end{document}
