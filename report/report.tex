% IEEE standard conference template; to be used with:
%   spconf.sty  - LaTeX style file, and
%   IEEEbib.bst - IEEE bibliography style file.
% --------------------------------------------------------------------------

\documentclass[letterpaper]{article}
\usepackage{spconf,amsmath,amssymb,graphicx}

% Example definitions.
% --------------------
% nice symbols for real and complex numbers
\newcommand{\R}[0]{\mathbb{R}}
\newcommand{\C}[0]{\mathbb{C}}

% bold paragraph titles
\newcommand{\mypar}[1]{{\bf #1.}}

% Title.
% ------
\title{Minimum Spanning Tree: a parallel approach}


% For example:
% ------------
%\address{School\\
%		 Department\\
%		 Address}
%
% Two addresses (uncomment and modify for two-address case).
% ----------------------------------------------------------
\twoauthors
  {Th. Cambier, R. Dang-Nhu, Th. Dardinier}
  {
	Ecole Polytechnique\\
  	{\small Route de Saclay, 91128 Palaiseau Cedex, FRANCE}\\
  	ETH Zürich, D-INFK\\
	{\small Rämistrasse 101, 8092 Zurich, SWITZERLAND}
  }
  {C. Trassoudaine\sthanks{The fourth author performed the work while at ETH Zürich}}
  {
	  IMT Atlantique,\\
	  {\small 655 Avenue du Technopôle, 29280 Plouzané, FRANCE}\\
	  EURECOM, Data Science dpt.,\\
	  {\small 450 route des Chappes, 06410 Biot, FRANCE}
  }


\begin{document}
%\ninept
%
\maketitle
%


\begin{abstract}

ABSTRACT

\end{abstract}

\section{Introduction}\label{sec:intro}

INTRO

\mypar{Motivation} 

What are we doing and why

What is done in the paper

\mypar{Related work} 

Some references, what do they do,
difference with us
make clear what our contribution is

\section{Background}\label{sec:background}

background information 

\mypar{MST}
MST problem

\mypar{MST algorithms}
MST algo and costs


\section{Method}\label{sec:yourmethod}

METHOD

\section{Experimental Results}\label{sec:exp}



\mypar{Experimental setup} platform (processor, frequency, maybe OS, maybe cache sizes)
as well as the compiler, version, and flags used. 

benchmarks (input sizes)

\mypar{Results}
classes of experiments answering to questions
compare code to external benchmarks.

\section{Conclusions}

What we did / why important

important results

next steps




% References should be produced using the bibtex program from suitable
% BiBTeX files (here: bibl_conf). The IEEEbib.bst bibliography
% style file from IEEE produces unsorted bibliography list.
% -------------------------------------------------------------------------
\bibliographystyle{IEEEbib}
\bibliography{bibl_conf}

\end{document}


